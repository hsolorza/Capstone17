\documentclass[letterpaper,draftclsnofoot,onecolumn,10 pt]{IEEEtran} 

%% Language and font encodings
\usepackage[english]{babel}
\usepackage[utf8x]{inputenc}
\usepackage[T1]{fontenc}

%% Sets page size and margins
\usepackage[a4paper,top=3cm,bottom=2cm,left=3cm,right=3cm,marginparwidth=1.75cm]{geometry}

%% Useful packages
\usepackage{amsmath}
\usepackage{fancyvrb,cprotect}
\usepackage{graphicx}
\usepackage[colorinlistoftodos]{todonotes}
\usepackage[colorlinks=true, allcolors=blue]{hyperref}

\title{iCreate: Problem Statement}
\author{Hannah Solorzano}

\begin{document}
\maketitle

\begin{abstract}
	Generative design can be described as a form finding process that has an evolutionary approach to design, which is the process of receiving the main goals of the design and then calculates all the possible ways to find a solution which meets the given criteria.  Dr. Rafaele De Amicis proposed a project that utilizes virtual reality to create a new tool for generative design that can be used when designing wood based architecture designs. There are three final results of this project: a program with an easy to use user interface that can render 3D designs, a website, and a robot that can build these designs.  Mike Premi of Intel will also be available to help the team members with software and hardware for the project.

\end{abstract}

\section{PROJECT DESCRITPTION}
The general description of the project is to create a virtual reality program in which a user can develop the early stages of a generative design. It is also noted that design should be created within a short time frame and with minimal errors. User design is another key factor in this project as there will be several interface options such as gestures and sketches. \par
	The three listed deliverables that are required for the completion of this project is a User Requirement and Task Analysis report, a System Architecture Report, and the 3D Modeling and Assembly program. In addition to these, the iCreate team will also provide a website that describes the project, it’s results, and pictures of the team that worked on it. This will be a public website that others can go to in order to get a better understanding of the project as well as observe the designs created by the program.\par
    Team members on this project are required to have experience in object oriented programming as the software that is used to develop virtual reality programs typically use \Verb!C++! or \Verb!C#!. Though these languages are similar, the software’s that use them have different capabilities regarding what they are capable of building and rendering. For example, the \Verb!C#! program, Unity, is known for its ease of use with a mild learning curve as well as expansive documentation and numerous forums. Unreal Engine, the \Verb!C++! program, is known for its ability to create realistic imagery and graphics and is used widely by professional game developers. For this project, the team is considering using Unity as it can program a 3D model program with less difficulty. The easiness of learning was also a key consideration as none of the team members have much experience with either Unity or Unreal Engine, and have no experience with programming virtual reality programs. Client De Amicis also requests that, as this project is about building 3D models, that at least one of the students has experience with graphics or 3D models.
\linebreak
\linebreak

\section{PROPOSED SOLUTION}

The solution for this project is to utilize Unity to create a program that offers geometric shapes, like bricks and triangles, which combines these shapes with an algorithm provided by the user and generates the 3D figure that transforms the shape to follow the curvature of the equation. The resulting shapes can be copy and pasted with a new algorithm to create an interesting architectural design. The user face for this program will either incorporate user gestures or an analog stick as the method of usability and the final decision will be determined by user study or depending on the hardware that is used for the final product. One of the deliverables is a website that can be shown to the general public so that they can see the research and project results. The website for this project will be created using tools like Wix or Bootstrap and will feature the problem statement, the research, final results, and the pictures of the team members that worked on the project.  The team has decided to use Wix rather than Bootstrap because we ran into the problem of acquiring a domain name. We did not want to have to pay for a domain name, and our ONID  accounts would not be able to host the website because our ONID accounts disappear after we graduate. Therefore, the team members will use Wix to develop and host the website.

\section{PERFORMANCE METRICS}

As stated previously, there are three deliverables listed in the project description: a report on User Requirements of Task Analysis, a System Architecture report, and the 3D Generative Design application. A fourth deliverable was given when the team met with Dr. De Amicis which was the website that gave information on the project. The two reports will be considered complete when they cover the entirety of the research and results that the team acquired throughout the duration of the project while the website will be considered complete when it is fully functioning with information on the project that is explained in a way that the general public can understand. The most important deliverable is the 3D design program. This program has many parts with which each part could be considered a small mini-deliverable. Firstly, the team was tasked with researching Unity and Unreal Engine, then use one of those to create a small program that can build a simple wall using a small brick and a user given equation. The purpose of this deliverable is to learn the basics of the tools that we are to use, as well as get a better understanding on how the main program will be designed. The next deliverable is the main program which creates a 3D generative design based off of the user given equation that allows for transformation and rotations. The third deliverable is not a concrete deliverable and is based off of it the team is able to finish the main program within a reasonable amount of time. This third deliverable is a robot which can take the generative design produced by the program and build it out of real blocks, such as LEGO’s. The completion of all these deliverables will signify the completion of the project as described by the client, Dr. De Amicis.


\end{document}
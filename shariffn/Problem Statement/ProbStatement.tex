\documentclass[letterpaper,10pt,onecolumn]{IEEEtran}
%draftclsnofoot

\usepackage{hyperref}
\usepackage{geometry}

\geometry{margin=0.75 in}

\setlength{\parskip}{1em}

\title{Problem Statement}
\author{Nabeel Shariff\\ ~ \\CS 461\\Fall 2017}

\begin{document}

\maketitle

\begin{abstract}

% Abstrct

\noindent
Our team, iCreate, seeks to develop a VR tool that allows users to create complex and beautiful structures made from algorithms created using generative design. This task will be accomplished under the guidance of Dr. Amicis.

\end{abstract}

\newpage

%----------------- PROBLEM -----------------%
\section{\textbf{Definition and Description of Problem }}

Today's software is capable of making almost anything happen, and as long as the user is satisfied with the product, our job is done. But today’s software, to become tomorrow’s software, needs to be researched and worked on in such a way that it produces a new state of the art software with intuitive user interface. The current problem is that given the popularity of VR, there is still a dearth of software that can aid with big projects in fields like architecture, engineering, etc. Most apps allow you to build small object in VR, but what about for designing skyscrapers, infrastructure, or an entire city in VR, quickly and efficiently? 

In order to allow the user to create complex structures using simple objects, algorithms need to be created that effectively translate an object several times in order to generate a complex structure.  

%----------------- PROPOSED SOLUTION -----------------%
%\noindent\rule{17.8cm}{0.4pt}
\section{\textbf{Proposed Solution}}

Our task is to create algorithms that take simple shapes and generate complex geometric structures in a 3D virtual environment.  The purpose of the project is to conduct research and develop these algorithms that are capable of translating any shape and generating a beautiful structure. Additionally, we will also be using a robot that assembles the structure with wood.  

We will be using either Unity or Unreal Engine as our VR environments, create a few simple algorithms first, then move onto more complex ones. For the robot, there is a library and/or api for Unity.  



%----------------- Performance Metrics -----------------%
%\noindent\rule{17.8cm}{0.4pt}
\section{\textbf{Performance Metrics}}

To start, we need to create a few simple algorithms that take a basic shape, like a brick, and duplicate and/or translate them several times, so that together all those bricks look like a complex structure like a roman water canal. 

Next, we will move onto more complex algorithms that can make beautiful and complex looking structure quickly using simple structures like blocks. 

After that, we want to use an api or library to allow a robot to use our algorithms to assemble these structures. We also want to implement these algorithms in an immersive virtual environment. Think Visual Studio for VR, just with way more power at the user's disposal. 

Finally, we will combine all our research and work to finalize a VR app that utilizes our algorithms. Additionally, we can also implement our ideas on a robot that assembles complex structures using simple wood objects. 
 

%\section*{Processes}
	%\subsection*{Windows}
	
\iffalse	
\begin{thebibliography}{4}
\bibitem{first}
F. Author. (year). 
\textit{title} 
. [Online]. Available: 
\\\url{url}
\end{thebibliography}
\fi

\end{document}




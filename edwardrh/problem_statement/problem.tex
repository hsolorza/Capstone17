\documentclass[letterpaper,10pt,onecolumn]{IEEEtran}
%draftclsnofoot

\usepackage{hyperref}
\usepackage{geometry}

\geometry{margin=0.75 in}

\setlength{\parskip}{1em}

\title{-\\ ~ \\ ~ \\ ~ \\ ~ \\ ~ \\ ~ \\ ~ \\ Problem Statement}
\author{Rhea Mae V. Edwards\\ ~ \\CS 461\\Fall 2017}

\begin{document}

\maketitle

\begin{abstract}

% Project abstract summarizing the entire document in 100-150 words

\noindent
This problem statement will describe our CS 361 senior capstone project, ICreate - Generative Design in Virtual Reality, and propose a handful of solutions to the requests of our group's client, assistant and research professor, Raffaele De Amicis. Overall, Professor Raffaele De Amicis would like to view our project more as a valuable learning opportunity throughout our group's progression within this academic 2017-2018 school year. The ICreate - Generative Design in Virtual Reality project focuses on generative designing, and learning how to implement the concept virtually, and possibly even programming a robot to build in such ways also. In addition, in order to complete such tasks, further research, practice, and testing in further understanding our outputs and solutions.

\end{abstract}

\newpage

%----------------- PROBLEM -----------------%
\section{\textbf{Problem}}

\iffalse
\noindent
Definition and description of the problem trying to solve
\par \noindent
Be sure to write this problem definition for a general but educated audience
\fi

\noindent
In the words of Professor Raffaele De Amicis, "designing with numbers," that is what our project can be described as in three simple words. The project ICreate - Generative Design in Virtual Reality proposes the problem or possibilities of creating a virtual reality program in regards to generative design, and being able to have that program fabricate such structures in actually reality. 

\noindent
According to our client, Raffaele De Amicis, our project would be treated more of an learning opportunity for the team at most. We are given scheduled tasks in the beginning in order to better understand and reach our end goals. The structure of the project is also very flexible in what our team is able to do in general. We are still given certain requirements in request from Professor Raffaele De Amicis, but how we complete such requirements, can vary based off our team's opinions and interests.

\noindent
Initially, Professor Raffaele De Amicis would like our team to become familiar with available software engines there are for our team to use. He has suggested Unity and Unreal, leaning towards the use of Unity, because of its more relatable application to our project as a whole. Individually, once we figure out which engine we would prefer to use, our first assignment from Professor De Amicis is to create a wall full of arches made out of brick-shaped elements. After completing this task, we will present our solution to Professor De Amicis, and continue with our progression in completing the end goal of our project of creating a virtual reality program of generative designing.

\noindent
Within the initial project description, the final outcomes requested are the following, stated as its deliverables:

\begin{itemize}
	\item User Requirements and Task Analysis - Report
	\item System Architecture - Report
	\item 3D Generative Design Based Scene Modeling and Assembly Application - Software
\end{itemize}

\noindent
Overall, tasks and assignments will be given to us from Professor De Amicis over the following month, in order to help us complete his end vision, and also providing our team with highly valuable educational experiences along the way. 

%----------------- PROPOSED SOLUTION -----------------%
%\noindent\rule{17.8cm}{0.4pt}
\section{\textbf{Proposed Solution}}

\iffalse
\noindent
We will first create some basic algorithms that translates an object in to complex shapes, than move onto more complex algorithms. We will use an API or library in Unity for a robot application.
\fi



%----------------- Performance Metrics -----------------%
%\noindent\rule{17.8cm}{0.4pt}
\section{\textbf{Performance Metrics}}

\iffalse
\noindent
Tell how will know when we completed the project
\par \noindent
Metrics help you and your client agree on what is successful completion of what the project looks like (e.g., faster, cheaper, easier to use, "a working prototype," a complete white paper with research results).
\fi

\noindent
As the year progress, our project further develops in complexity.

%\section*{Processes}
	%\subsection*{Windows}

\end{document}


\documentclass[letterpaper,10pt,onecolumn]{IEEEtran}
%draftclsnofoot

\usepackage{hyperref}
\usepackage{geometry}

\geometry{margin=0.75 in}

\setlength{\parskip}{1em}

\title{-\\ ~ \\ ~ \\ ~ \\ ~ \\ ~ \\ ~ \\ ~ \\ ~ \\ ~ \\ Problem Statement}
\author{Rhea Mae V. Edwards\\ ~ \\CS 461\\Fall 2017}

\begin{document}

\maketitle

\begin{abstract}

% Project abstract summarizing the entire document in 100-150 words

\noindent
This problem statement will describe our CS 361 senior capstone project, ICreate - Generative Design in Virtual Reality, and propose a handful of solutions to the requests of our group's client, assistant and research professor, Raffaele De Amicis. Overall, Professor Raffaele De Amicis would like to view our project more as a valuable learning opportunity throughout our group's progression within this academic 2017-2018 school year. The ICreate - Generative Design in Virtual Reality project focuses on generative designing, and learning how to implement the concept virtually, and possibly even programming a robot to build in such ways also. In addition, in order to complete such tasks, further research, practice, and testing in further understanding our outputs and solutions.

\end{abstract}

\newpage

%----------------- PROBLEM -----------------%
\section{\textbf{Problem}}

\iffalse
\noindent
Definition and description of the problem trying to solve
\par \noindent
Be sure to write this problem definition for a general but educated audience
\fi

In the words of Professor Raffaele De Amicis, "designing with numbers," that is what our project can be described in three simple words. The project ICreate - Generative Design in Virtual Reality proposes the problem or possibilities of creating a virtual reality program in regards to generative design, and being able to have that program realistically fabricate such structures. 

%----------------- PROPOSED SOLUTION -----------------%
%\noindent\rule{17.8cm}{0.4pt}
\section{\textbf{Proposed Solution}}

\iffalse
\noindent
We will first create some basic algorithms that translates an object in to complex shapes, than move onto more complex algorithms. We will use an API or library in Unity for a robot application.
\fi



%----------------- Performance Metrics -----------------%
%\noindent\rule{17.8cm}{0.4pt}
\section{\textbf{Performance Metrics}}

\iffalse
\noindent
Tell how will know when we completed the project
\par \noindent
Metrics help you and your client agree on what is successful completion of what the project looks like (e.g., faster, cheaper, easier to use, "a working prototype," a complete white paper with research results).
\fi

As the year progress, our project further develops in complexity.

%\section*{Processes}
	%\subsection*{Windows}
	
\iffalse	
\begin{thebibliography}{4}
\bibitem{first}
F. Author. (year). 
\textit{title} 
. [Online]. Available: 
\\\url{url}
\end{thebibliography}
\fi

\end{document}

